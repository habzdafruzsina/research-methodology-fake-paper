\documentclass{article}
\usepackage{graphicx} % Required for inserting images
\usepackage{draftwatermark}

\SetWatermarkText{Fictional}
\SetWatermarkFontSize{4cm}
\SetWatermarkScale{4.4}

\title{Automatic improvement of the location of web elements in automated test cases with the help of an FTPK tool}
\author{test-automation group}
\date{April 2023}

\begin{document}

\maketitle

\section{Introduction}

Software testing is an essential phase in the software development life cycle (SDLC). It helps to ensure that the software product meets the specified requirements and functions as expected. Test automation has gained considerable attention due to its potential to reduce the testing time and improve the quality of the software product. However, maintaining test scripts can be challenging, especially when the program changes. Existing tools can have costly runtime, and they can be inaccurate, and may not have the ability to handle the problem of test oracle. Therefore, there is a need for a new approach to test automation that can address these issues.

Manual testing has been the traditional method of testing software applications. However, with the increasing complexity of software applications, manual testing has become time-consuming, expensive, and prone to human error. Automation testing has emerged as a viable alternative to manual testing, providing developers with the ability to execute tests automatically, quickly, and with minimal human intervention.

Automation testing is crucial in ensuring that software applications meet the desired quality standards. Automation testing provides several advantages over manual testing, including faster test execution, increased test coverage, and better accuracy. However, automation testing also has its disadvantages, including the need for technical expertise, higher initial costs, and the risk of false positives and false negatives.

There are several types of automation testing, including functional testing, regression testing, load testing, and performance testing. Each type of automation testing has its specific use case, and developers can choose the appropriate type of automation testing based on the application's requirements.

DevOps is a software development methodology that emphasizes collaboration and integration between developers and operations teams. Automation testing plays a critical role in the DevOps methodology, as it enables developers to integrate testing into their development pipeline, facilitating the continuous delivery of software applications.

In conclusion, automation testing has become an essential aspect of modern software development. Its advantages in terms of speed, accuracy, and test coverage make it a compelling alternative to manual testing. The various types of automation testing and its connection to DevOps further highlight the importance of automation testing in software development. This research paper aims to provide an in-depth analysis of automation testing and its significance in the software development life cycle.

In this research paper, we propose a new approach to test automation using web element localization to make more robust test cases for GUI testing. We mainly focused on web elements, but this approach can be used also on other GUI structures, such as Android apps, or Windows desktop applications. The main idea is to relocalize web elements by their properties and tell if the change in the code was the desired, correct behaviour or not. 

Our research aims to address this challenge by introducing a new web element localization method that is capable of finding major differences between code states and providing solutions to adjust automated test cases accordingly.

The proposed web element localization method is designed to identify web elements based on their unique attributes, such as ID, class, name, or XPath, across different code states. By comparing these attributes, the method is able to detect major changes in the web page's structure and provide solutions to adjust automated test cases. These solutions may include modifying test scripts or updating web element locators to ensure that the automated test cases continue to function as expected.

This approach can help to stop rerunning our webelement relocalization algorithm, which produces an optimized solution. It can improve the accuracy and reliability of automated test cases by ensuring that they continue to function as expected.

By providing solutions to adjust automated test cases, our proposed method can help organizations to achieve faster software delivery cycles while maintaining high levels of quality and reliability. We named our tool FTPK after the researchers.

The following questions will guide our research:

\begin{itemize}
\item How can we relocalize webelements by their properties?
\item How many times should the approach search?
\item Can this approach help to reduce the cost of test automation?
\item Can this approach improve the accuracy of test results?
\item Can this approach solve the test oracle problem?
\item Is this approach better, than Similo, or Dalia Alamleh's approach alone?
\end{itemize}

In the next sections of this research paper, we will describe our tool in detail, present the results of our experiments, and discuss the implications of our findings.

\maketitle

\section{Related work}

Related work in this area includes WATER, WATERFALL, COLOR, ROBULA, ATA-QW, SIDEREAL, Leotta's Multi-Locator (LML), Fuzzy-DEMATEL, Neuro-Fuzzy Logic (NFL), Dalia Alamleh's approach (DAA) and AI in test automation. These approaches have various strengths and weaknesses, and they focus on different aspects of test automation.

Talk a bit about these in further details...

Our solution is based on the Similo and DAA. We tried to combine them in a way to use the strength of both, and eliminate the weeknesses. Similo utilizes the triangulation of multiple locator information to identify correct GUI elements (web elements in this study). The approach is shown to be more effective at finding elements than the baseline solution and efficient enough for practical use. However, defining a suitable value for the threshold is non-trivial. If the threshold is set too high, that might eliminate valid matches, and if it is set too low, incorrect matches may be chosen due to the aforementioned synchronization challenge. So we try to esteem this value by using the DAA, which is able to tell if the code was correctly changed or not. DAA proposed a test automation which utilizes an intelligent decision-making algorithm known as fuzzy logic by using Fuzzy set theory which classifies the inputs to predict the output. This approach can predict any possible results for a combination of two inputs. According to Dalia Alamleh's results; the Fuzzy inference system seems to be a great artificial intelligent approach that can provide a test oracle for functional testing applied on web applications.

\maketitle

\section{Methodology}

process:
run tests
get failed tests
classify failed tests by the not found web elem.
run DAA fuzzy logic to distinguish correct and incorrect code changes 
put aside the tests for incorrect code changes -> FAILED  

1st run similo on one test from each group with (correct changes)
rerun failed tests

*n

show results to human



\maketitle

\section{Result}

(robustness, performance)

\maketitle

\section{Discussion}


\maketitle

\section{Conclusions and future work}


\maketitle

\section{Acknowledgements}

We would like to express our deepest gratitude towards the Eötvös Loránd University for providing us with the resources and facilities needed to conduct this research. Without the support of the university, this study would not have been possible.

We also extend our sincere thanks to the two anonymous reviewers who provided valuable feedback and constructive criticism that helped improve the quality of this paper. Their insights and suggestions have greatly contributed to the final version of this manuscript.

Finally, we would like to thank all the participants who generously gave their time and energy to take part in this study. Their cooperation and willingness to share their experiences have been invaluable to our research.

\maketitle

\section{References}

\end{document}
